\documentclass{article}

% Language setting
% Replace `english' with e.g. `spanish' to change the document language
\usepackage[russian]{babel}


% Set page size and margins
% Replace `letterpaper' with `a4paper' for UK/EU standard size
\usepackage[a4paper,top=1cm,bottom=1.1cm,left=2cm,right=2cm,marginparwidth=1.75cm]{geometry}

% Useful packages
\usepackage{amsmath}
\usepackage{amsthm}
\usepackage{graphicx}
\usepackage{wrapfig}
\documentclass{article}
\usepackage[colorlinks=true, allcolors=blue]{hyperref}

\title{Лекция по алгему}
\author{Анонимы ПМИ-1х}
\date{11.05.22}
\begin{document}


\maketitle
\Large

\section{Свойства}
\textbf{Свойство 1.} \newline
Если \(\lambda_{1}, \lambda_{2}, ... ,  \lambda_{n}\) - это собственные числа матрицы \(n\)-го порядка \(A\),\newline то \(f(\lambda_{1}), f(\lambda_{2}), ... ,  f(\lambda_{n})\) -  собственные значения от \(f(A)\) \newline
Без доказательства\newline
\newline
\textbf{Свойство 2.}\newline
Если \(A\) и \(B\) подобные, причем \(B = P^{-1}  A  P\), то матрицы \(f(A)\) и \(f(B)\) подобные, и матрица \(B\) преобразует \(f(A)\) в \(f(B)\).\newline
\(B = P^{-1}AP = \mapsto f(B) = P^{-1} f(A) P\)\newline
\(J = P^{-1}AP \mapsto A = P J P^{-1} \mapsto f(A) = Pf(J)P^{-1}\)\newline
\(\text{.....рассмотреть доказательство в Гандмахере (теории матриц)}\)\newline
\newline
\textbf{Свойство 3}\newline
Если \(A\) - квазидиагональная матрица\newline\newline
A = 
\begin{pmatrix}
 A_{1} & 0 & ... & 0 \\
 0 & A_{2} & ... & 0 \\
 ... & ... & ... & ... \\
 0&  0& ... & A_{n} \\
\end{pmatrix} \mapsto f(A) = \begin{pmatrix}
 f(A_{1}) & 0 & ... & 0 \\
 0 & f(A_{2}) & ... & 0 \\
 ... & ... & ... & ... \\
 0&  0& ... & f(A_{n}) \\
\end{pmatrix}\newline\newline
\(\text{......доказательство там же}\)\newline
\newline

\textbf{Пример 3: Оператор простой структуры}\newline\newline
A = P \begin{pmatrix}
 \lambda_{1} & 0 & ... & 0 \\
 0 & \lambda_{2} & ... & 0 \\
 ... & ... & ... & ... \\
 0&  0& ... & \lambda_{n} \\
\end{pmatrix} P^{-1}
\newline \newline \newline
f(A) = P \begin{pmatrix}
 f(\lambda_{1}) & 0 & ... & 0 \\
 0 & f(\lambda_{2}) & ... & 0 \\
 ... & ... & ... & ... \\
 0&  0& ... & f(\lambda_{n}) \\
\end{pmatrix} P^{-1}
\newline

\textbf{Пример 4:} Более общая картина \newline
\(J = diag(J_{\nu_{1}}(\lambda_{1}), ... , J_{\nu_{u}}(\lambda_{u}))\)\newline
По свойству (3) получаем:\newline
\(f(J) = diag(f(J_{\nu_{1}}(\lambda_{1})), ... , f(J_{\nu_{u}}(\lambda_{u})))\)\newline
Ссылаясь на пример (2) доказываем. \newline
\newline
\section{Интерполяционный многочлен Лагранжа Сильвестра}\newline
\(r(\lambda) \mapsto f(A) = r (A)\)\newline\newline
\(\psi(\lambda) = E_{n} (\lambda) = \prod_{i = 1}^{S} (\lambda - \lambda_{i})^{m_{i}}\)\newline\newline
\(\sum_{i = 1}^{S} m_{i} - m = deg\textbf{ }\psi(\lambda)\)\newline\newline
Представим функцию \(\frac{r(\lambda)}{\psi(\lambda)}\), являющейся правильной дробью, в виде суммы простых дробей\newline
\(\sum_{i = 1}^{S} \left [ \frac{\alpha_{i1}}{(\lambda - \lambda_{i})^{m_{i}}} + \frac{\alpha_{i2}}{(\lambda - \lambda_{i})^{m_{i}-1}} + 
\frac{\alpha_{i, m_{i}}}{(\lambda - \lambda_{i})}\right ]\textbf{(4)}\)\newline
где \(\alpha_{i,m_{i}} = \overline{1, S}\)\newline

Для определения числителя простых дробей умножим обе части равенства (4) на 
\((\lambda - \lambda_{k})^{m_{k}}\) и обозначим через \newline

\(\psi^{2}(\alpha) = \frac{\psi(\lambda)}{(\lambda - \lambda_{k})^{m_{k}}} \mapsto 
\alpha_{k1} + \alpha_{k2}(\lambda - \lambda_{k}) + ... + \alpha_{km_{k}}(\lambda - \lambda_{k})^{m_{k}-1} + \rho (\lambda) (\lambda-\lambda_{k})^{m_{k}}\newline\textbf{(5)}\)\newline
\newline
\(\rho(\alpha)\) - рациональная функция, которая не обращается в бесконечность при \(\lambda = \lambda_{k}\)\newline

Найдем коэффициенты \(\alpha_{k1}, \alpha_{k2}, ..., \alpha_{ki}\)\newline
\newline
\(\alpha_{k1} = \frac{r(\lambda)}{\psi_{k}(\alpha)}_{\left ( \lambda = \lambda_{k} \right )}\)
\(\alpha_{k2} = (\frac{r(\lambda)}{\psi_{2}(\alpha)})^{'}_{\left ( \lambda = \lambda_{k} \right )}\) \textbf{(6)}\newline
\newline
Формулы (6) показывают, что числители в правой части равенства (4) выражаются через значения многочлена \(r(\lambda)\) на спектре матрицы \(A\)\newline
\newline
А эти значения нам известны. Они равны соответствующим значениям функции \(f(\lambda)\) и её производных, поэтому можно обобщить \newline
\newline
\(\alpha_{ij} = \frac{1}{(j-1)!} \left [ \frac{f(\lambda)}{\psi^{'}(\lambda)} \right]^{(j-1)}_{\lambda = \lambda_{i}}  \)\textbf{(7)}\newline
\newline
После того, как все \(\lambda\) найдены, мы определяем \(r(\lambda)\) по следующей формуле, которая получается из (4) умножением на \(\psi(\lambda)\)\newline
\newline
\(r(\lambda) = \sum _{i = 1} ^{S} \left [ \alpha_{i1} + \alpha_{i2}(\lambda - \lambda_{i}) + ... + \alpha_{i, m_{i}}(\lambda - \lambda_{i})^{m_{i}-1}\right] \psi^{'}(\lambda) \textbf{(8)}\newline\)
\newline
\textbf{Замечание:}\newline
Выражение в квадратных скобках, стоящее в качестве \(\psi(\lambda) = \sum_{i = 1}^{n}\) разложения Тейлора по степеням \((\lambda - \lambda_{i})\) функции \(\frac{r(\lambda)}{\psi(\lambda)} (?)\)\newline

\textbf{УПРАЖНЕНИЕ:}\newline
Получить интерполяционный многочлен для\newline
\(\psi(\lambda) = (\lambda - \lambda_{1})^2 (\lambda - \lambda_{2})^3\)\newline
\newline
\section{Основная формула}\newline
Вернемся к формуле (8) для \(r(\lambda)\), подставляя в нее выражение (7) для коэффициентов \(\alpha\) (4), объединяя члены, которые содержат одно и то же значение \(f(\lambda)\) и какой-либо ее производной мы представим \(r(\lambda)\) в виде:\newline
\[r(\lambda) =  
\sum _{i = 1} ^{S} \left [ f(\lambda_{i}) \varphi_{i1}(\lambda) + f^{'}(\lambda_{i}) \varphi_{i2}(\lambda) + ... + f^{(m_{i} - 1)}(\lambda_{i}) \varphi_{i,m_{i}}(\lambda)\right] \textbf{(9)}\]
\newline

\(\lambda_{i1}\) - полином \(\lambda < m\)\newline
\newline
Эти полиномы определяются заданием \(\psi(\lambda)\) и не зависят от выбора \(f(\lambda)\)\newline
\newline
Число таких полиномов равно числу значений \(f(\lambda)\) на спектре матрицы \(A\), т.е равно \(m\).\newline
\newline
Из формулы (9) сразу же следует основная формула:\newline
\[f(A) = \sum _{i = 1} ^{S} \left [ f(\lambda_{i}) Z_{i1} + f^{'}(\lambda_{i}) Z_{i2} + ... + f^{m_{i} - 1}(\lambda_{i}) Z_{i, m_{i}}(\lambda)\right] \textbf{(10)}\]
\newline
\(Z_{ij} = \varphi _{ij}\) - компоненты матрицы \(А\)\newline
\(f(\lambda)\) - линейно независимая система\newline
и еще одно задание: разложить интерполяционный многочлен из \newlineУПРАЖНЕНИЯ по этой формуле}\newline
\textit{The final lection.}


\end{document}