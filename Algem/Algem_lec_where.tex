\documentclass{article}

% Language setting
% Replace `english' with e.g. `spanish' to change the document language
\usepackage[russian]{babel}

% Set page size and margins
% Replace `letterpaper' with `a4paper' for UK/EU standard size
\usepackage[letterpaper,top=2cm,bottom=2cm,left=3cm,right=3cm,marginparwidth=1.75cm]{geometry}

% Useful packages
\usepackage{amsmath}
\usepackage{graphicx}
\usepackage[colorlinks=true, allcolors=blue]{hyperref}

\title{Лекция по алгему}
\author{Анонимы ПМИ-1х}

\begin{document}
\maketitle
/// - разобраться на досуге под чашку горячего чая
\section{Кратчайшее расстояние между двумя скрещивающимися прямыми}
Тройка векторов линейно независима \newline
Минимизируем квадрат длинны:\newline
\[\omega ^2 = (x_{1} + at - x_{2} - bu, x_1 + at - x^2 - bu ) = ... = (x_1 - x_2, x_1 - x_2) + 2(a_1x_1-x^2)t - 2(b_1x_1 - x_2) = 2(a, b) tu + (a,a)t^2 + (b,b)u^2\]
\[\begin{matrix}\frac{\delta \omega ^2}{\delta t} = 0 
\\ 
\frac{\delta \omega ^2}{\delta u} = 0 
\end{matrix} \mapsto
\ \begin{bmatrix}
(a,a) & -(a,b) \\ 
-(a,b) & (b,b)
\end{bmatrix}
\begin{bmatrix}
t\\ 
u
\end{bmatrix}
=
\begin{bmatrix}
-(a_1x_1 - x_2)\\ 
(bx_2-x_1)
\end{bmatrix}
\textbf{(14)}\]
\(a, b \) - линейно независимы \(\Rightarrow\) система не вырождена\newline
Решение ур (14):\newline
\[t = \frac{\begin{vmatrix}
-(a, x_1-x_2) & -(a,b)\\ 
(b, x_1- x_2) & (b,b)
\end{vmatrix}}{
\ \begin{vmatrix}
(a,a) & -(a,b) \\ 
-(a,b) & (b,b)
\end{vmatrix}}
\]\
\newline
\[
u = \frac{
\begin{vmatrix}
(a,a) & (a,x_1-x_2)\\ 
(a,b) & (b, x_1-x_2) 
\end{vmatrix}
}{\begin{vmatrix}
(a,a) & -(a,b) \\ 
-(a,b) & (b,b)
\end{vmatrix}}\text{  }\textbf{(15)}\]
Построим матрицу матрицу Гессе (2-х прозводных):
\[\begin{bmatrix}\frac{\delta \omega ^2}{\delta t^2} = 0 
\\ 
\frac{\delta \omega ^2}{\delta u^2} = 0 
\end{bmatrix} = 
\ \begin{bmatrix}
2(a,a) & -2(a,b) \\ 
2(a,b) & 2(b,b)
\end{bmatrix}\]
\[w^2 = \frac{\begin{vmatrix}
(x_1-x_2,x_1-x_2) & (a, x_1-x_2) & (b,x_1-x_2)\\ 
(a, x_1-x_2)  & (a,a) & (a,b)\\ 
(b,x_1-x_2) & (a,b) & (b,b)
\end{vmatrix}}{\begin{vmatrix}
(a,a) & -(a,b) \\ 
-(a,b) & (b,b)
\end{vmatrix}}\]
/// Привести вычисления
\section{Общий перпендикуляр 2-х прямых}
Требуя чтобы отрезок, соединяющий произвольные точки двух данных прямых (10) был перпендикулярен направленным векторам \(a, b\) - обеих прямых
\[(x_1-x_2+at-bu, a) = (x_1-x_2,a) + (a,a)t - (b,a)u = 0
\newline
(x_1-x_2+at-bu, b) = (x_1-x_2,b) + (a,b)t - (b,b)u = 0\]
Мы получили усл (14), откуда находим, что значение t и удовлетворяющее этому условию совпадают с (15) - это совпадение показывает, что основания общего перпендикуляра 2-х прямых совпадают с теми точками этих прямых, расстояние между которыми минимально. 
\section{Геометрия плоскостей}
Определение:\newline
Будем называть m-мерной плоскостью аффинного пространства размерности n или m - плоскостью. Множество всех точек этого пространства, полученных из одной его точки всеми переносами, векторы которых компланарны и принадлежат одному линейному пространству. \newline
Поскольку векторы этих переносов имеют вид \(a_\alpha t\alpha\), где за \(t\alpha\) принимают вещ. значение\newline
Радиус точек m плоскости имеют вид:\newline
\[x = x_0 + \sum_{\alpha = 1}^{m}a_\alpha t\alpha\ \textbf{(1)}\]
m = 0 - плоскость представляет собой точку. 
\newline
Для того, чтобы 2 плоскости были параллельными необходимо, чтобы \(\left \{ a_{\alpha} \right \}\ |_{\alpha = 1}^{m} \sim \left \{ b_{\alpha} \right \}_{\alpha = 1}^{m}\)
\section{Уравнение m по m+1 точке}
Если заданы \(M+1\) точка \( M_0(x_0),  M_1(x_1),  ...  M_{\alpha}(x_{\alpha})\)   \newline
Если векторы \(M_0, ..., M_{\alpha}\) линейно независимы, то эти точки определяют единственную m плоскость, проходящую через них. \newline
Векторное представление:  (усл 2 пропущено)
\[x = x_0 + \sum_{\alpha = 1}^{m}(x_{\alpha} - x_0) t\alpha\ \textbf{(3)}\]
Исключение: m = n - 1\newline
m - векторное пространство n\newline
m - гиперплоскость (или просто плоскость)\newline
\section{Векторное уравнение плоскости}
Если мы умножим обе части ур (1) на вектор \(u\), перпендикулярный всем векторам (нормальный вектор), то получим \( (u,x) - (u,x_0) + \sum_{\alpha = 1}^{n-1}(u_{\alpha} a_{\alpha}) t\alpha\ = (u,x) + v = 0 (5)\), где \(v = -(u, x_0)\)
\newline
Координатное уравнение плоскости:\newline
Если мы выбираем ортонгормировнный базис, то\newline

\[\left \{ CV^n, i= \bar{{1,n}} \right \} \mapsto u =  \sum_{i=1}^{n}u_{i}e_{i} \newline
x = \sum_{i=1}^{n}x_{i}e_{i} \mapsto \sum_{i=1}^{n-1}u_{i}x_{i} + v = 0  \text{  } \textbf{(6)}\]
/// \(Ax + By + Cz + D = 0\) 

\section{Уравнение плоскости и вектора нормали}
/// Разобраться ссылаясь на (6) + ресурсы 
\section{Основная теорема о плоскости}
Уравнение всякой плоскости аффинных координатах является линейным уравнением и наоборот. \newline
Всякое линейное уравнение в линейной плоскости является аффинным.
/// Да. 
\section{Уравнение плоскости по точке и направляющем векторам}
Отношение 1:\newline
[данные удалены, на доске было что-то подправлено]
\[(x-x_0) + t_1a_1 + ... + t_{n-1}a_{n-1}\]
Равная нулю линейная комбинация векторов с ненулевым хотя бы одним коэффициентом \(\Rightarrow \) система линейно независима. \newline
Развернем через определитель
\[\begin{Vmatrix}
x_1 - x_{01} & a_{11} & ... & a_{n-1,1}\\ 
x_2 - x_{0n} & a_{12} & ... & a_{n-1,2}\\ 
... & ... &  ...&  ...\\ 
x_n - x_{0n} & a_{1n} & ... & a_{n-1,n}
\end{Vmatrix} = 0 \text{  }\textbf{(9)}\]
Уравнение плоскости по n - точкам:
\[\begin{Vmatrix}
x_1 - x_{01} & x_{11} - x_{01} & ... & x_{n-1,1} - x_0\\ 
x_2 - x_{0n} & x_{12} - x_{02} & ... & x_{n-1,2} - x_2\\ 
... & ... &  ...&  ...\\ 
x_n - x_{0n} & x_{1n} - x_{0n} & ... & x_{n-1,n} - x_n
\end{Vmatrix}\]
\section{Угол между двумя плоскостями}
Будем называть тот из углов между векторами нормали b - плоскостей, который \(\leq \frac{\pi}{2}\)(по ур 12)\newline
\[cos\phi = \frac{\left \| u_1,u_2 \right \|}{\left \| u_2 \right \|\left \| u_2 \right \|}\text{  } \textbf{(12)}\]
\section{Расстояние от точки до плоскости}
Будем называть расстоянием от точки до плоскости наименьшее расстояние от данной точки до точек плоскости, т.к мин. расстояние от данной точки до точек всякой прямой, лежащей на плоскости, явл. расстоянием от данной точки до основания перпендикуляра, опущенного на r - расстояние от точки до основания перпендикуляра, опущенного из него на плоскость. \newline
Найдем расстояния от точки \(\tilde{m}\) с радиус-вектором x до плоскости (5):\newline
Уравнение перпендикуляра, опущенного из точки \(\tilde{m}\) на плоскость \textbf{\(x = \tilde{x} + ut\) (13) }
\newline
Подставим (13) в (5) и найдем t:\newline
\[(u, x)tv = (u, \tilde{x}) + (u, u)t + v = 0 \text{  } \textbf{(14)}\]
Поскольку расстояние от точки \(\tilde{m}\) c радиус-вектором \(\tilde{r}\) о произвольной точки плоскости есть \(\omega\), то:
\[\tilde{M}\omega = \left \| x - \tilde{x} \right \| = \left \| ut
 \right \|\]
 /// Расстояние от точки до плоскости представить в виде (14)

 
\end{document}
