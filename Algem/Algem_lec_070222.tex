\documentclass{article}

% Language setting
% Replace `english' with e.g. `spanish' to change the document language
\usepackage[russian]{babel}

% Set page size and margins
% Replace `letterpaper' with`a4paper' for UK/EU standard size
\usepackage[letterpaper,top=1cm,bottom=1cm,left=1cm,right=1cm,marginparwidth=1.75cm]{geometry}

% Useful packages
\usepackage{amsmath}
\usepackage{graphicx}
\usepackage[colorlinks=true, allcolors=blue]{hyperref}

\title{Лекция по алггему}
\author{Квадратичные формы}

\begin{document}
\maketitle

\section{То, что осталось с прошлого семестра (есть опечатки)}
Определение \newline
Пусть  \(A(x,y)\) - симметричная билинейная форма \newline
\(f(A(x,x)) \), которая получается путем подстановки \(A(x,x) \) называется квадратичной формой. При этом \( A(x,y)\) называется билинейной полярной формой. \newline 
Таким образом каждой симметричной билинейной форме соответствует одна квадратичная форма. \newline
По квадратичной форме билинейную полярную восстановить нельзя.\newline 

Рассмотрим:\newline
\(A(x+y; x+y) = A(x,x) + A(x,y) + A(y, x) + A(y,y) \Rightarrow 
\)
\(A(x,y) = (A(x+y; x+y) - A(x, x) - A(y,y)) * \frac{1}{2}\)

Определение\newline
Квадратичная форма - положительно определенная, если скалярное произведение есть билинейная форма, соответственно определена билинейной форме и наоборот (??).\newline
Любая такая форма может быть принята за скалярное произведение.\newline
\[A(x,x) \geq 0{ , }\forall x \in \theta_L\]\newline
Евклидово пространство можно определить следующим образом:\newline
Евклидовым пространством называется действительное линейное пространство, в котором выбрана какая-либо фиксированная положительно опр. квадратичная форма \(A(x,x)\) значение \(A(x,y)\) соответствующее ей билинейной формы считает следом скалярным произведением векторов \( x,y\).\newline

\section{Настоящее}
Покажем, как привести квадратичную форму к сумме квадратов, то есть выбрать такой базис, в котом квадратичная форма имеет наиболее простой (канонический вид)\newline
\[A(x,x) = \lambda_{1}\alpha_1^2 + ... + \lambda_{n}\alpha_n^2\]\newline
Пусть в некотором базисе \(f_{1}...f_{n}\) существует веществуенное линейное пространство \(L\) , квадратичная форма имеет вид:\newline
\[A(x,x) =\sum_{i = 1}^{n}\sum_{k = 1}^{n}\alpha_{ik}*\beta{i}*\beta{k}\]\newline
Преобразуем базис так, чтобы в \(A(x,x)\) остались только квадраты и пропали смешанные произведения. \newline
При этом мы будем фиксировать только изменение координат, не записывая изменения базисов. \newline

Пусть в \(A(x,x)\) хотя бы 1 из коэффициентов \(a_{ik}\) (коэффициентов при \(B_{i}^{2}\)) отличен от нуля. \newline
Если это не так, то квадратичная форма содержит хотя бы одно смешанное произведение. 
Например:\newline
\[2a_{12}*B_{1}*B_{2} = 2a_{12}(B_{1}^{`}+B_{2}^{`})(B_{1}^{`}-B_{2}^{`})=2a_{12}(B_{1}^{`})^2-2a_{12}(B_{2}^{`})^2\]\newline

Полученные квадратные члены ни с чем не сократятся. \newline
Будем считать, что в квадратичной форме имеются ненулевые коэффициенты и \(a_{11} \neq 0\)\newline
\[\alpha_{11}\beta_1^2+2\alpha_{12}\beta_{1}\beta_{2}+...+2\alpha_{1n}\beta_{n}\beta_{n}=\frac{1}{\alpha_{11}}(\alpha_{11}\beta_{1}+ ... + \alpha_{1n}\beta_{n})^2 - \beta\]\newline
Дополним эту сумму до полного квадрата\newline

 \(\beta \) - члены, которые содержат квадраты и попарные произведения \(\alpha_{12}\beta_{2} ... \alpha_{1n}\beta_{n}\)\newline
 Подставляя последнее выражение в \(A(x,x)\) получим:\newline
 \[A(x,x) = \frac{1}{\alpha_{1n}}(\alpha_{11}\beta_{1} + ... + \alpha_{1n} \beta_{11})^2\]
 Делаем замену, полагая:\newline
 \[A(x,x) = \frac{1}{\alpha_{11}}\beta_{1}^{\star}+\sum_{i=2}^{n}\sum_{k=2}^{n}\alpha_{ik}\beta_2^{\star}\beta_n^{\star}\]\newline
 НО, мы понижаем размерность, следовательно \(\beta_{2}^{\star}\) и последующие пересчитываются\newline
 \[\beta_1^{\star} = \alpha_{11}\beta_{1} + ... + \alpha_{1n}\beta_{n}
\\\beta_2^{\star} = \beta_2
\\\beta_n^{\star} = \beta_n\]
Продолжая этот процесс в конечномерном пространстве в итоге получим (1), где \(\alpha_{1}...\alpha_{n}\) - координаты векторов в новом базисе.\newline
Доказана теорема:\newline
Пусть в n-мерном вещественное пространстве  L задана производная квадратичная форма A(x,x), тогда в L существует базис \(e_{1}...e_{n}\) такой, что эта квадратичная форма имеет вид:\newline

 \[A(x,x) = \lambda_{1}\alpha_1^2 + ... + \lambda_{n}\alpha_n^2\]
\(\alpha\) - координаты вектора x в базисе \(e_{1}...e_{n}\).\newline

\section{Приведение квадратичной суммы к сумме квадратов методом Якоби}
 Дадим еще один способ построения базиса, в котором квадратичная форма приводится к форме квадратов. В отличии от предыдущего метода получим формулу, выражающую искомый базис \(e_1...e_n\) непосредственно через исходный базис.\newline
 Пусть имеем симметричную билинейную форму \(A(x, y)\) с матрицей \(A\).\newline
 \[(A)_{ik}=a_{ik}=A(f_{i}; f_{k})\]\newline
Предположим, что все главные миноры матрицы \(A\) отличны друг от нуля.\newline
\[\triangle_1=a_{11} \neq 0 {;}
\\ \triangle_2= \begin{bmatrix}a_{11} & a_{12} \\a_{21} & a_{22} \end{bmatrix} \neq 0{;}
\\ \triangle_n = \begin{bmatrix}a_{11} & a_{12} & {...} & a_{1n} \\{...} & {...} & {...} & {...} \\ a_{1n} & {...} & {...} & a_{nn} \end{bmatrix}\neq 0\]
Необходимо найти такой базис \(e_1...e_n\), чтобы \(A(e_{1}...e_{n}) = 0, i \neq k\eqno(2)\).\newline
Процесс, с помощью которого это было сделано напоминает процесс ортогонализации скалярного произведения \((x, y)\). Берется \((x,x)\),  поэтому будем искать по Грамму-Шмидту.\newline
\[e_{1}=a_{11}f_1
\\e_2 = a_{21}f_1 + a_{22}f_2
\\e_n = a_{n1}f_1 + a_{n2}f_2 + a_{nn}f_n\eqno(3)\eqno(1)\]
Заметим, что если \newline
\[A(e_{k};f_{i}) = 0, i = \overline{1, k-1}\Rightarrow A(e_{k}; e_{i}) = 0, i = \overline{1, k-1} \eqno(4)\]
Действительно:\newline
\[A(e_{k}; e_{i}) = A(e_{k}, a_{i1}f_{i} + ... +a_{ii}f_i) = a_{i1}A(e_{1}; e_{1}) + ... + a_{ii}A(e_{k}; e_{k}), i = \overline{1,k-1}\]
В силу симметричности билинейной формы:\newline
\[(A(e_{k}; e_{i}) =  0 , i>k)\]\newline
Наша задача свелась к определению коэффициентов \(a_{k1}...a_{kk}\) таким образом, чтобы вектор \(e_{k} = a_{k1}f_{1}+ a_{k2}f_{2} + ... +  a_{kk}f_{k}\) удовлетворял условиям линейного пространства размерностью (k - (k-1)) с точностью до числа k (5)
\[A(e_{k}; f_{k}) = 1\eqno(5)\]
Подставим в соотношения 4 и 5 выражение (ну то, перед цифрой 4)\newline
\[\begin{cases}A(e_{k}; f_{1}) = a_{11}\alpha_{k1}+a_{21}\alpha_{k2} + ... + a_{k1}\alpha_{kk} = 0 
\\ 
A(e_{k}; f_{2}) = a_{12}\alpha_{k1}+a_{22}\alpha_{k2} + ... + a_{k2}\alpha_{kk} = 0
\\
...............................................
\\
A(e_{k}; f_{k}) = a_{1k}\alpha_{k1}+a_{2k}\alpha_{k2} + ... + a_{kk}\alpha_{kk} = 0 \end{cases}\eqno(6)\]

Определитель СЛАУ (6) по условию отличен от нуля. Решение 6 существует и единственно.\newline 
Найдем матрицу (\(\beta)_{ik} = b_{ik} = (6) = A(e_{i}; e_{k})\).\newline
По построению базиса \(e\) можем сказать, что \(b_{ik} = 0, i \neq k\).\newline
Вычислим \(b_{ik}\):\newline
\[A(e_k; e_k) = A(e_k, a_{k1}f_1 + ... + a_{kk}f_k) = a_{k1}A(e_k; f_1) + ... + a_{kk}A(e_k; f_k) = a_{kk} = \frac{\triangle_{k-1}}{\triangle_{k}}= \triangle_{0} = 1\]\newline
Число \(a_{kk}\) можно найти из системы (6). Согласно правилу Крамера последние три равенства в выражении. \newline
Доказана теорема 1.\newline


\end{document}